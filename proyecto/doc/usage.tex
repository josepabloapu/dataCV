%% LaTeX Document
%% Author: Jose Pablo Apú, Jose Carlos Campos, Jorge Soto
%% Document Class
\documentclass[11pt]{article}
%%*************************************************************************
%% Most used packages
\usepackage{graphicx}				%%include figures
\usepackage[utf8]{inputenc}			%%include latin characters
\usepackage[english]{babel}			%%spanish
\usepackage[vmargin=4cm,			%%margins
	    tmargin=3cm,
	    hmargin=2cm,
	    letterpaper]{geometry}
\usepackage{color}				%%include colors
\usepackage{fancyhdr}				%%use fancy headers and footes
\usepackage{framed}				%%framed boxes
\usepackage{hyperref}				%%make hyperlinks
\usepackage{listings}				%%
%%*************************************************************************
%% Other settings
\definecolor{shadecolor}{rgb}{1,0.8,0.3}	%%boxes color
\setlength{\parindent}{0pt}			%%indemptation
\pagestyle{fancy}				%%needed to include fancy styles
\renewcommand{\headrulewidth}{0.25mm}		%%linewidth of the headers
\pagenumbering{gobble}				%%remove the page number
\graphicspath{{../../multimedia/imagenes/}}	%%multimedia path
\definecolor{shadecolor}{RGB}{250,250,250}	%%boxes color
\lstset{
	basicstyle=\ttfamily,
	keywordstyle=\color{black},
	commentstyle=\color{black},
	stringstyle=\color{black},
	tabsize=2,
	backgroundcolor=\color{shadecolor}}
%%*************************************************************************
%% Fancy Header
%% EIE's template
\fancyhead[L]{\includegraphics[scale=0.15]{ucr}}
\fancyhead[C]{UNIVERSIDAD DE COSTA RICA\\
	      ESCUELA DE INGENIERÍA ELECTRICA\\
	      ESTRUCTURAS ABSTRACTAS DE DATOS Y ALGORITMOS\\
	      PARA INGENIERIA - IE0217\\
	      PROYECTO I - GUÍA DE USO}
\fancyhead[R]{\includegraphics[scale=0.4]{eie}}
\pagestyle{fancy}
\setlength{\headheight}{60pt}
%%*************************************************************************
\begin{document}

\begin{center}
{ \huge \bfseries Data Computing Visualization Library }\\[0.2cm]
{ Jose Carlos Campos - Jose Pablo Apu - Jorge Soto }\\[0.2cm]
\rule{\linewidth}{0.25mm}
\end{center}

Primero se crea el costructor de la clase\\
\begin{lstlisting}
	Mysqplot(string);
\end{lstlisting}	
En necesario que el parámetro del contructor sea el nombre de la tabla. Luego se crea la conexión con la base de datos\\
\begin{lstlisting}
	bool conn(const char*,const char*,const char*,const char*); 
\end{lstlisting}
Donde se introduce la base de datos, el servidor, el usuario y la clave respectivamente. Luego se utiliza alguna de estas funciones:

\begin{lstlisting}

	float mean(const char*);
	
	float variance(const char*);
	
	float standard_deviation(const char*);
	
	double kolmogorov_test_uniform(const char*,float,float);

	bool histogram(const char*,int);
	
	bool gaussian_distribution(const char*);
	
	bool jitterplot(const char*, int);
		
	bool scatterplot(const char*, const char*);
	
	bool kde(const char*,float=1.0);
	 
	bool pdf(const char*,bool=false);
	 
	bool cdf(const char*);

\end{lstlisting}
  	
Finalmente para liberar espacio de la ram, se incluye el destructor.
\begin{lstlisting}
	~Mysqplot();
\end{lstlisting}
%%*************************************************************************	
\end{document}
