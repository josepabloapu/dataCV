%% LaTeX Document
%% Author: Jose Pablo Apú, Jose Carlos Campos, Jorge Soto
%% Document Class
\documentclass[11pt]{article}
%%*************************************************************************
%% Most used packages
\usepackage{graphicx}				%%include figures
\usepackage[utf8]{inputenc}			%%include latin characters
\usepackage[english]{babel}			%%spanish
\usepackage[vmargin=4cm,			%%margins
	    tmargin=3cm,
	    hmargin=2cm,
	    letterpaper]{geometry}
\usepackage{color}				%%include colors
\usepackage{fancyhdr}				%%use fancy headers and footes
\usepackage{framed}				%%framed boxes
\usepackage{hyperref}				%%make hyperlinks
\usepackage{listings}				%%
%%*************************************************************************
%% Other settings
\definecolor{shadecolor}{rgb}{1,0.8,0.3}	%%boxes color
\setlength{\parindent}{0pt}			%%indemptation
\pagestyle{fancy}				%%needed to include fancy styles
\renewcommand{\headrulewidth}{0.25mm}		%%linewidth of the headers
\pagenumbering{gobble}				%%remove the page number
\graphicspath{{../../multimedia/imagenes/}}	%%multimedia path
\definecolor{shadecolor}{RGB}{250,250,250}	%%boxes color
\lstset{
	basicstyle=\ttfamily,
	keywordstyle=\color{black},
	commentstyle=\color{black},
	stringstyle=\color{black},
	tabsize=2,
	backgroundcolor=\color{shadecolor}}
%%*************************************************************************
%% Fancy Header
%% EIE's template
\fancyhead[L]{\includegraphics[scale=0.15]{ucr}}
\fancyhead[C]{UNIVERSIDAD DE COSTA RICA\\
	      ESCUELA DE INGENIERÍA ELECTRICA\\
	      ESTRUCTURAS ABSTRACTAS DE DATOS Y ALGORITMOS\\
	      PARA INGENIERIA - IE0217\\
	      PROYECTO I - GUÍA DE INSTALACIÓN}
\fancyhead[R]{\includegraphics[scale=0.4]{eie}}
\pagestyle{fancy}
\setlength{\headheight}{60pt}
%%*************************************************************************
\begin{document}

\begin{center}
{ \huge \bfseries Data Computing Visualization Library }\\[0.2cm]
{ Jose Carlos Campos - Jose Pablo Apu - Jorge Soto }\\[0.2cm]
\rule{\linewidth}{0.25mm}
\end{center}

\subsection*{Introducción}
Estas instrucciones son basadas en la distribución de debian, sirve tanto para la estable como para la versión testing, pero sirve para otras distribuciones. Mysqplot es un wrapper de mysql++ y gnuplot-cpp, por lo que hay que estar seguros de haber instalado ambas previamente. 

\subsection*{Dependencias}
\begin{itemize}
\item mysql++\\
\begin{itemize}
\item download the source files from \url{http://tangentsoft.net/mysql++/}
\item make sure you have installed libmysql++-dev
\item to know the path of the mysql-lib you can make a locate after doing an updatedb
\end{itemize}
\begin{lstlisting}
./configure --with-mysql-lib=/PATH/OF/THE/MYSQL-LIB
make
sudo make install
\end{lstlisting}
\item gnuplot-cpp\\

\begin{lstlisting}
sudo apt-get install gnuplot-x11
\end{lstlisting}
\end{itemize}
%% References
\begin{thebibliography}{10}
\bibitem{tangent}\url{http://tangentsoft.net/mysql++/}
\bibitem{opengl}\url{http://www.opengl.org/}
\bibitem{sfml}\url{http://www.sfml-dev.org/}
\bibitem{oreilly}Steele, J., \textit{Why data visualization matters}, 
		\url{http://strata.oreilly.com/2012/02/why-data-visualization-matters.html}
\bibitem{dundas}Dundas Data Visualization, Inc., \textit{Making Business Decisions Easier with Data Visualizations}
		\url{http://www.dundas.com/discover/article/making-business-decisions-easier-with-data-visualizations/}
\end{thebibliography}
%%*************************************************************************	
\end{document}
