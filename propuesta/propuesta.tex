%% LaTeX Document
%% Author: Jose Pablo Apú
%% Document Class
\documentclass[11pt]{article}
%%*************************************************************************
%% Most used packages
\usepackage{graphicx}				%%include figures
\usepackage[utf8]{inputenc}			%%include latin characters
\usepackage[spanish]{babel}			%%spanish
\usepackage[vmargin=4cm,			%%margins
	    tmargin=3cm,
	    hmargin=2cm,
	    letterpaper]{geometry}
\usepackage{color}				%%include colors
\usepackage{fancyhdr}				%%use fancy headers and footes
%\usepackage{framed}				%%framed boxes
\usepackage{hyperref}				%%make hyperlinks
%%*************************************************************************
%% Other settings
\definecolor{shadecolor}{rgb}{1,0.8,0.3}	%%boxes color
\setlength{\parindent}{0pt}			%%indemptation
\pagestyle{fancy}				%%needed to include fancy styles
\renewcommand{\headrulewidth}{0.25mm}		%%linewidth of the headers
\pagenumbering{gobble}				%%remove the page number
\graphicspath{{../multimedia/imagenes/}}	%%multimedia path
%%*************************************************************************
%% Fancy Header
%% EIE's template
\fancyhead[L]{\includegraphics[scale=0.15]{ucr}}
\fancyhead[C]{UNIVERSIDAD DE COSTA RICA\\
	      ESCUELA DE INGENIERÍA ELECTRICA\\
	      ESTRUCTURAS ABSTRACTAS DE DATOS Y ALGORITMOS\\
	      PARA INGENIERIA - IE0217\\
	      PROYECTO I - PROPUESTA}
\fancyhead[R]{\includegraphics[scale=0.4]{eie}}
\pagestyle{fancy}
\setlength{\headheight}{60pt}
%%*************************************************************************
\begin{document}

\begin{center}
{ \huge \bfseries Data Computing Visualization Library }\\[0.2cm]
{ Jose Campos - Jose Pablo Apu - Jorge Soto }\\[0.2cm]
\rule{\linewidth}{0.25mm}
\end{center}

\subsection*{Introducción}
Con el fin de brindar una mayor facilidad para visualizar los datos, la idea es crear una libreria, en c++, que tome los datos a traves de una base de datos y los despliegue en pantalla de una manera grafica. Para lograr dicha tarea se utilizara mysql como manejador de bases de datos, utilizando la libreria mysql++, y para visualizar los datos de manera grafica se piensa utilizar la libreria llamada plot++. 

\subsection*{Justificación}
Justificacion

\subsubsection*{Objetivo General}
objetivo general

\subsubsection*{Objetivos Específicos}
\begin{itemize}
\item objetivo 1
\item objetivo 2
\end{itemize}
\subsection*{Metodología}

%% References
\begin{thebibliography}{10}
\bibitem{label}Ref 1
\bibitem{label}\url{http://www.url.com}
\end{thebibliography}
%%*************************************************************************	
\end{document}