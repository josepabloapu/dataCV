%% LaTeX Document
%% Author: Jose Pablo Apú
%% Document Class
\documentclass[11pt]{article}
%%*************************************************************************
%% Most used packages
\usepackage{graphicx}				%%include figures
\usepackage[utf8]{inputenc}			%%include latin characters
\usepackage[english]{babel}			%%spanish
\usepackage[vmargin=4cm,			%%margins
	    tmargin=3cm,
	    hmargin=2cm,
	    letterpaper]{geometry}
\usepackage{color}				%%include colors
\usepackage{fancyhdr}				%%use fancy headers and footes
%\usepackage{framed}				%%framed boxes
\usepackage{hyperref}				%%make hyperlinks
%%*************************************************************************
%% Other settings
\definecolor{shadecolor}{rgb}{1,0.8,0.3}	%%boxes color
\setlength{\parindent}{0pt}			%%indemptation
\pagestyle{fancy}				%%needed to include fancy styles
\renewcommand{\headrulewidth}{0.25mm}		%%linewidth of the headers
\pagenumbering{gobble}				%%remove the page number
\graphicspath{{../multimedia/imagenes/}}	%%multimedia path
%%*************************************************************************
%% Fancy Header
%% EIE's template
\fancyhead[L]{\includegraphics[scale=0.15]{ucr}}
\fancyhead[C]{UNIVERSIDAD DE COSTA RICA\\
	      ESCUELA DE INGENIERÍA ELECTRICA\\
	      ESTRUCTURAS ABSTRACTAS DE DATOS Y ALGORITMOS\\
	      PARA INGENIERIA - IE0217\\
	      PROYECTO I - PROPUESTA}
\fancyhead[R]{\includegraphics[scale=0.4]{eie}}
\pagestyle{fancy}
\setlength{\headheight}{60pt}
%%*************************************************************************
\begin{document}

\begin{center}
{ \huge \bfseries Data Computing Visualization Library }\\[0.2cm]
{ Jose Carlos Campos - Jose Pablo Apu - Jorge Soto }\\[0.2cm]
\rule{\linewidth}{0.25mm}
\end{center}

\subsection*{Introducción}
Con el fin de brindar una mayor facilidad para visualizar los datos, la idea es crear una libreria, en c++, que tome los datos a traves de una base de datos y los despliegue en pantalla de una manera grafica. Para lograr dicha tarea se utilizara mysql como manejador de bases de datos, utilizando la libreria mysql++, y para visualizar los datos de manera grafica se piensa utilizar la libreria llamada plot++. Además se pretende implementar un set de herramientas para obtener datos estadísticos de los datos obtenidos de la base de datos. 

\subsection*{Justificación}
La librería está diseñada para que cualquier usuario de MySql pueda utilizarla. La visualización de datos permite un mejor análisis y una mayor comprensión de los mismos. En una base de datos, la información es prácticamente números y letras únicamente. Al graficar dicha información, se puede entender de una mejor manera el significado de los datos.

En el análisis de datos, se deben tomar decisiones acerca de los mismos, de las que dependen clientes, productos y beneficios. Una manera eficiente de hacerlo es con la visualización de datos. Además, permite una relación más directa del público con una compañía y sus expertos. %%% microresumen de la pagina strata.oreilly

En el nivel emprasarial, se debe hacer uso de herramientas de visualización de datos. La cantidad de datos es proporcional al tamaño de la empresa. Teniendo muchos datos, es más dificil encontrar patrones y relaciones para la correcta toma de decisiones. Se necesita elegir de manera rápida y precisa las decisiones de negocios para que la empresa permanezca competitiva. %%% dundas

\subsubsection*{Objetivo General}
Diseñar una biblioteca que permita la visualización de la información en una base de datos.

\subsubsection*{Objetivos Específicos}
\begin{itemize}
\item Importar la información de una base de datos que utiliza el manejador MySql.
\item Crear un módulo que tome los datos importados desde MySql y los convierta en información compatible para la librería que grafica.
\item Generar un histograma a partir de los datos importados.
\item Generar un gráfico de dispersión a partir de los datos importados.
\item Implementar un conjunto de herramientas que permitan un análisis estadístico de los datos.
\end{itemize}
\subsection*{Metodología}
Mediante la utilización de librerías ya creadas, como lo son mysql++, sfml, y chplot o bien openGL, se creará herramientas para facilitar al programador visualizar los datos.
%% References
\begin{thebibliography}{10}
\bibitem{label}\url{http://www.dundas.com/discover/article/making-business-decisions-easier-with-data-visualizations/}
\bibitem{label}\url{http://www.opengl.org/}
\bibitem{label}\url{http://www.sfml-dev.org/}
\bibitem{label}\url{http://strata.oreilly.com/2012/02/why-data-visualization-matters.html}
\bibitem{label}\url{http://tangentsoft.net/mysql++/}
\end{thebibliography}
%%*************************************************************************	
\end{document}